\documentclass[fleqn]{article}
\usepackage[T2A]{fontenc}
\usepackage[utf8]{inputenc}
\usepackage[russian]{babel}
\pagenumbering{gobble}

\begin{document}
\begin{large}
    \section*{Касательная}
1) Касательная к графику функции $f(x)=3x^3+x^2-14x+2$ параллельна прямой $y=-3x-5$ и касается графика в точке $(1; -8)$,\newline
Определите значение производной функции $f(x)$ в точке касания.\newline\newline
2) Касательная к графику функции $f(x)=2x^3-9x+1$ параллельна прямой $y=-3x-2$ и касается графика в точке $(1; -6)$,\newline
Определите значение производной функции $f(x)$ в точке касания.\newline\newline
3) Касательная к графику функции $f(x)=5x^3-2x^2-6x-4$ параллельна прямой $y=5x$ и касается графика в точке $(1; -7)$,\newline
Определите значение производной функции $f(x)$ в точке касания.\newline\newline
4) Касательная к графику функции $f(x)=-2x^3+2x^2+6x-2$ параллельна прямой $y=4x+4$ и касается графика в точке $(1; 4)$,\newline
Определите значение производной функции $f(x)$ в точке касания.\newline\newline
5) Касательная к графику функции $f(x)=5x^3-4x^2-11x-2$ параллельна прямой $y=-4x+5$ и касается графика в точке $(1; -12)$,\newline
Определите значение производной функции $f(x)$ в точке касания.\newline\newline
    \section*{Ответы: }
\[1)\: -3\]
\[2)\: -3\]
\[3)\: 5\]
\[4)\: 4\]
\[5)\: -4\]
\end{large}
\end{document}
